\documentclass[12pt]{article}
\usepackage{latexsym}
\usepackage{amsfonts}
\usepackage{amsmath}
\usepackage{xcolor}



%%Formatting
\topmargin -.2in
\textheight 9.2in
\evensidemargin 0in
\oddsidemargin 0in
\textwidth 6.5in
\parskip .1in

\pagestyle{empty}

\begin{document}

\hrule
\vspace{.2cm}

{\Large \noindent Math 112 Honors
\hfill
B\'ela Bajnok}

\vspace{.3cm}
\hrule

{\Large \noindent 
Exam 5
\hfill
Fall 2020}

\vspace{.3cm}
\hrule



\noindent NAME:  Quan Nguyen


\noindent \hrulefill\rule{0pt}{4pt}

\noindent Write and sign the full Honor Pledge here:

\vspace{2mm}

I affirm that I have upheld the highest principles of honesty and integrity in my academic work and have not witnessed a violation of the Honor Code. \par

Quan Nguyen

\vspace{8mm}

\noindent \hrulefill\rule{0pt}{4pt}

\noindent {\bf General instructions -- Please read!}

\begin{itemize}
  
\item The purpose of this exam is to give you an opportunity to explore a complex and challenging question, gain a fuller view of calculus and its applications, and develop some creative writing, problem solving, and research skills.

\item  {\bf All your assertions must be completely and fully justified.} At the same time, you should aim to be as concise as possible; avoid overly lengthy arguments and unnecessary components.  Your grade will be based on both mathematical accuracy and clarity of presentation.

\item 
Your finished exam should read as an article, consisting of complete sentences, thorough explanations, and exhibit correct grammar and punctuation.

\item I encourage you to prepare your exam using LaTeX.  However, you may use instead other typesetting programs that you like, and you may use hand-writing or hand-drawing for some parts of your exam.  In any case, {\bf the final version that you submit must be in PDF format.}  

\item It is acceptable (and even encouraged) to discuss the exams with other students in your class or with the PLA.  However,  {\bf  you must individually write up all parts of your exams.}

\item {\bf You may use the text, your notes, and your homework, but no other sources.}
 
\item You must write out a complete, honest, and detailed acknowledgment of all assistance you received and all resources you used (including other people) on all written work submitted for a grade.

\item Submit your exam to me by email at bbajnok@gettysburg.edu by the deadline announced in class.







\end{itemize}

\noindent {\bf Good luck!}


\newpage

\begin{center}

{\Large Solitaire Army} 

\end{center}

\vspace*{.2in}


\noindent  The game Solitaire Army is a particular version of Peg Solitaire; it was discovered by the British mathematician John Conway in 1961.   Peg Solitaire is played on a board that has holes arranged in a rectangular grid-like fashion (like squares on an infinite chess-board).  Each hole can hold one peg.  A move consists of a jump by one peg over another peg which is next to it horizontally or vertically (but not diagonally); the peg jumped over will then be removed from the board.  Each move therefore reduces the total number of pegs by one.

In the game Solitaire Army, the board is an infinite plane where one horizontal line is distinguished; it's called the demarcation line.  At the start of the game, all pegs are on one side of the demarcation line.  

1.  Verify that it is possible to send one peg forward 1, 2, 3, or 4 holes into the other side of the demarcation line.  

2.  Use geometric series to prove that it is impossible to get to further than that.  



\newpage

\begin{center}
\underline {\Large {My work} }
\end{center}

% http://www.maths.dur.ac.uk/~bmjg46/conwaysarmy.pdf \par
% http://recmath.org/pegsolitaire/army/index.html \par
% https://www.chiark.greenend.org.uk/~sgtatham/solarmy/ \par

%Infinite series: http://sites.science.oregonstate.edu/math/home/programs/undergrad/CalculusQuestStudyGuides/SandS/SeriesTests/test_summary.html?fbclid=IwAR1kenZT96DIzuUSKPynIELd1xKLWbMUmUNXtZ9cVDjFx7uCw4bpb06z05U


Assuming that:
\begin{itemize}
    \item The cell I want to move the peg to is $1$.
    \item $i$ (a positive number) is the value of the cell containing it.
    \item For every cell that is further than 1, the value of that cell is multiplied by $i$.
\end{itemize}

\section{Moving the peg to the 1st row:}
\begin{tabular}{|l|c|c|c|c|c|c|c|c|c|c|r|}
    \hline
    $i^2$ \\
    \hline
    $i$ \\
    \hline
    $1$ \\
    \hline
\end{tabular}

\begin{itemize}
    \item From the table above, I have $i+i^2=1$
    
    The value of $i$ (using the equation above): \par
    
    \begin{align*}
        i+i^2&=1\\
        \Longleftrightarrow
        i^2+i-1&=0\\
    \end{align*}

    $i^2+i-1=0$ has the general quadratic equation: $ax^2+bx+c=0$, which means that $a=1$, $b=1$, and $c=-1$. The quadratic formula that is used to find the value of $i$:

    \begin{align*}
        & i = \frac{-b\pm \sqrt{b^2-4ac}}{2a} \\
        \Longleftrightarrow\;
        & i = \frac{-1\pm \sqrt{\left(-1\right)^2 -4\cdot 1 \cdot (-1)}}{2\cdot 1}\\
        \Longleftrightarrow\;
        & i_1 = \frac{-1\pm\sqrt{5}}{2}\\
        \Longleftrightarrow\;
        & i_1 = -1.618034
        \quad \lor\quad i_2 = 0.618034
    \end{align*}


    From what I assumed at the beginning:
    \begin{align*}
        &
        \begin{cases}
            1=i^2+i \\
            i>0
        \end{cases}\\
        \Longrightarrow &
        \begin{cases}
            1>i\\
            1>0
        \end{cases}\\
        \Longrightarrow & 1>i>0\\
        \Longrightarrow & i= i_2=0.618033
    \end{align*}

    Because $0<i<1$, $i^2$ must be smaller than $i$:
    $$0<i^2<i<1$$

    Similarly, $i^n$ with $n\to \infty$ will result in:
    $$0<i^n<i^{n-1}<\dots<i^3<i^2<i<1$$
\end{itemize}


    
\section{Moving the peg to the 2nd row:}

I can move the pegs to the 2nd row with 4 pegs: 1 peg $i^2$, 2 pegs $i^3$, and 1 peg $i^4$:


\begin{tabular}{|l|c|c|c|c|c|c|c|c|c|c|r|}
    \hline
    $i^3$ &  & \\
    \hline
    $i^2$ & $i^3$ & $i^4$  \\
    \hline
    &  &  \\
    \hline
    $1$ &  &  \\
    \hline
\end{tabular}
$\longrightarrow$
\begin{tabular}{|l|c|c|c|c|c|c|c|c|c|c|r|}
    \hline
    &  & \\
    \hline
    & $i^3$ & $i^4$  \\
    \hline
    \color{red}{$i$} &  &  \\
    \hline
    $1$ &  &  \\
    \hline
\end{tabular}
$\longrightarrow$
\begin{tabular}{|l|c|c|c|c|c|c|c|c|c|c|r|}
    \hline
    \\
    \hline
    \color{red}{$i^2$} \\
    \hline
    $i$ \\
    \hline
    $1$ \\
    \hline
\end{tabular}

*Note: Red pegs are new moves.


\noindent Checking:
    \begin{align*}
        i^2+2i^3+i^4 &= i^2+i^3+i^3+i^4\\
        &= i \left(i+i^2 \right)+i^2 \left(i+i^2 \right)\\
        &=i+ i^2\\
        &= 1
    \end{align*}


\noindent *As I can convert $i^2+i^3=i$ and $i^3+i^4=i^2$, I can now use the formula: $i^n=i^{n+1}+i^{n+2}$ for the following problems.
    
\section{Moving the peg to the 3rd row:}
\begin{tabular}{|l|c|c|c|c|c|c|c|c|c|c|r|}
    \hline
    & & & $i^6$ & & & \\
    \hline
    & & $i^6$ & \color{blue}{$i^5$} & $i^6$ & & \\
    \hline
    & $i^6$ & \color{blue}{$i^5$} & \color{black}{$i^4$} & \color{blue}{$i^5$} & \color{black}{$i^6$} & \\
    \hline
    $i^6$ & \color{blue}{$i^5$} & \color{black}{$i^4$} & \color{blue}{$i^3$} & \color{black}{$i^4$} & \color{blue}{$i^5$} & $i^6$ \\
    \hline
    & & & & & &\\
    \hline
    & & & & & &\\
    \hline
    & & & $1$ & & &\\
    \hline
\end{tabular}

\noindent From the table above, $1$ is on the 3rd row, so $i^3$ is the nearest cell. Therefore, I can only have 1 peg $i^3$, 3 pegs $i^4$, 5 pegs $i^5$, 7 pegs $i^6$, and so on.

\begin{align*}
    1&=i+i^2\\
    &= i^2+i^3+i^3+i^4\\
    &= \left(i^3+i^4\right)+2\left(i^4+i^5\right)+i^4\\
    &=i^3+ \left(i^5+i^6\right)+2\left(i^4+i^5\right)+i^4\\
    &\text{(I must convert 1 $i^4$ to keep only 3 $i^4$)}\\
    &=i^3+ 3i^4+ 3i^5+ i^6 \quad\text{(sum $=8$ pegs)}\\
\end{align*}

Now I place the pegs on the table to check if I am correct.


\begin{tabular}{|l|c|c|c|c|c|c|c|c|c|c|r|}
    \hline
    & & $i^4$ & $i^5$ & $i^6$ \\
    \hline
    $i^5$ & $i^4$ & $i^3$ & $i^4$ & $i^5$ \\
    \hline
    & & & &\\
    \hline
    & & & &\\
    \hline
    & & $1$ & &\\
    \hline
\end{tabular}
$\longrightarrow$
\begin{tabular}{|l|c|c|c|c|c|c|c|c|c|c|r|}
    \hline
    & & & $i^5$ & $i^6$ \\
    \hline
    $i^5$ & $i^4$ & & $i^4$ & $i^5$ \\
    \hline
    & & \color{red}{$i^2$} & &\\
    \hline
    & & & &\\
    \hline
    & & $1$ & &\\
    \hline
\end{tabular}
$\longrightarrow$
\begin{tabular}{|l|c|c|c|c|c|c|c|c|c|c|r|}
    \hline
    & & \color{red}{$i^4$} \\
    \hline
    $i^5$ & $i^4$ & \color{red}{$i^3$}  \\
    \hline
    & & $i^2$ \\
    \hline
    & & \\
    \hline
    & & $1$ \\
    \hline
\end{tabular}
$\longrightarrow$
\begin{tabular}{|l|c|c|c|c|c|c|c|c|c|c|r|}
    \hline
    & & $i^4$ \\
    \hline
    $i^5$ & $i^4$ & \\
    \hline
    & & \\
    \hline
    & & \color{red}{$i$}\\
    \hline
    & & $1$ \\
    \hline
\end{tabular}

$\longrightarrow$
\begin{tabular}{|l|c|c|c|c|c|c|c|c|c|c|r|}
    \hline
    $i^4$ \\
    \hline
    \color{red}{$i^3$}  \\
    \hline
    \\
    \hline
    $i$\\
    \hline
    $1$ \\
    \hline
\end{tabular}
$\longrightarrow$
\begin{tabular}{|l|c|c|c|c|c|c|c|c|c|c|r|}
    \hline
    \color{red}{$i^2$}\\
    \hline
    $i$\\
    \hline
    $1$ \\
    \hline
\end{tabular}

*Note: Red pegs are new moves.




\section{Moving the peg to the 4th row:}

\noindent Similar to the case above, the cell containing $i^4$ is the nearest cell as $1$ is on the 4th row. Therefore, I can only have 1 peg $i^4$, 3 pegs $i^5$, 5 pegs $i^6$, 7 pegs $i^7$, and so on.

\begin{align*}
    & 1 \\
    =& i + i^2 \\
    =& i^2+i^3 +i^3+i^4\\
    =& i^4+i^3+i^4+ 2i^4+ 2i^5\\
    =& i^4+ 2i^5+ 3i^4+ i^3\\
    =& i^4+ 3i^5+ 4i^4\\
    =& i^4+ 3i^5+ 4i^5+ 4i^6\\
    =& i^4+ 3i^5+ 5i^6+ 3i^6+ 4i^7\\    
    =& i^4+ 3i^5+ 5i^6+ 6i^7+ i^7+ 3i^8\\
    =& i^4+ 3i^5+ 5i^6+ 6i^7+ 4i^8+ i^9 \;\text{(sum $=20$ egs)}\\
\end{align*}

Now I place the pegs on the table to check if I am correct.


\begin{tabular}{|l|c|c|c|c|c|c|c|c|c|c|r|}
    \hline
    & & $i^9$ & & $i^7$ & & \\
    \hline
    & & $i^8$ & $i^7$ & $i^6$ & $i^7$ & $i^8$\\
    \hline
    & $i^8$ & $i^7$ & $i^6$ & $i^5$ & $i^6$ & $i^7$ \\
    \hline
    $i^8$ & $i^7$ & $i^6$ & $i^5$ & $i^4$ & $i^5$ & $i^6$\\
    \hline
    & & & & & & \\
    \hline
    & & & & & & \\
    \hline
    & & & & & & \\
    \hline
    & & & & $1$ & & \\
    \hline
\end{tabular}
$\longrightarrow$
\begin{tabular}{|l|c|c|c|c|c|c|c|c|c|c|r|}
    \hline
    & & $i^9$ & & $i^7$ & & \\
    \hline
    & & $i^8$ & $i^7$ & $i^6$ & $i^7$ & $i^8$\\
    \hline
    & $i^8$ & $i^7$ & $i^6$ & & $i^6$ & $i^7$ \\
    \hline
    $i^8$ & $i^7$ & $i^6$ & $i^5$ & & $i^5$ & $i^6$\\
    \hline
    & & & & \color{red}{$i^3$} & & \\
    \hline
    & & & & & & \\
    \hline
    & & & & & & \\
    \hline
    & & & & $1$ & & \\
    \hline
\end{tabular}

$\longrightarrow$
\begin{tabular}{|l|c|c|c|c|c|c|c|c|c|c|r|}
    \hline
    & & $i^9$ & & $i^7$ & & \\
    \hline
    & & $i^8$ & $i^7$ & $i^6$ & $i^7$ & $i^8$\\
    \hline
    & $i^8$ & $i^7$ & $i^6$ & & &\\
    \hline
    $i^8$ & $i^7$ & & & \color{red}{$i^4$} & &\\
    \hline
    & & & & $i^3$ & \color{red}{$i^4$} & \color{red}{$i^5$} \\
    \hline
    & & & & & & \\
    \hline
    & & & & & & \\
    \hline
    & & & & $1$ & & \\
    \hline
\end{tabular}
$\longrightarrow$
\begin{tabular}{|l|c|c|c|c|c|c|c|c|c|c|r|}
    \hline
    & $i^9$ & & & & \\
    \hline
    & $i^8$ & & & $i^7$ & $i^8$\\
    \hline
    $i^8$ & $i^7$ & & \color{red}{$i^5$} & &\\
    \hline
    & \color{red}{$i^6$} & \color{red}{$i^5$} & & &\\
    \hline
    & & & & $i^4$ & $i^5$ \\
    \hline
    & & & \color{red}{$i^2$} & & \\
    \hline
    & & & & & \\
    \hline
    & & & $1$ & & \\
    \hline
\end{tabular}
$\longrightarrow$
\begin{tabular}{|l|c|c|c|c|c|c|c|c|c|c|r|}
    \hline
    $i^9$ & &\\
    \hline
    $i^8$ & & \color{red}{$i^6$}\\
    \hline
    & \color{red}{$i^6$} & $i^5$\\
    \hline
    $i^6$ & $i^5$ &\\
    \hline
    & & \color{red}{$i^3$}\\
    \hline
    & & $i^2$\\
    \hline
    & &\\
    \hline
    & & $1$\\
    \hline
\end{tabular}

$\longrightarrow$
\begin{tabular}{|l|c|c|c|c|c|c|c|c|c|c|r|}
    \hline
    \color{red}{$i^7$} & $i^6$ &\\
    \hline
    $i^6$ & $i^5$ & \color{red}{$i^4$}\\
    \hline
    & &\\
    \hline
    & &\\
    \hline
    & & \color{red}{$i$}\\
    \hline
    & & $1$\\
    \hline
\end{tabular}
$\longrightarrow$
\begin{tabular}{|l|c|c|c|c|c|c|c|c|c|c|r|}
    \hline
    & & \color{red}{$i^5$}\\
    \hline
    $i^6$ & $i^5$ & $i^4$\\
    \hline
    & &\\
    \hline
    & &\\
    \hline
    & & $i$\\
    \hline
    & & $1$\\
    \hline
\end{tabular}
$\longrightarrow$
\begin{tabular}{|l|c|c|c|c|c|c|c|c|c|c|r|}
    \hline
    $i^6$ & $i^5$ &\\
    \hline
    & & \color{red}{$i^3$}\\
    \hline
    & &\\
    \hline
    & & $i$\\
    \hline
    & & $1$\\
    \hline
\end{tabular}
$\longrightarrow$
\begin{tabular}{|l|c|c|c|c|c|c|c|c|c|c|r|}
    \hline
    \color{red}{$i^4$}\\
    \hline
    $i^3$\\
    \hline
    \\
    \hline
    $i$\\
    \hline
    $1$\\
    \hline
\end{tabular}
$\longrightarrow$
\begin{tabular}{|l|c|c|c|c|c|c|c|c|c|c|r|}
    \hline
    \color{red}{$i^2$}\\
    \hline
    $i$\\
    \hline
    $1$\\
    \hline
\end{tabular}

*Note: Red pegs are new moves.




    
    
    
    
        
\section{Moving the peg to the 5th row:}
    \begin{tabular}{|l|c|c|c|c|c|c|c|c|c|c|r|}
        (\dots) & (E') & (D') & (C') & (B') & (A) & (B) & (C) & (D) & (E) & (\dots)\\
        \hline
        \dots & \dots & \dots & \dots & \dots & \dots & \dots & \dots & \dots & \dots & \dots \\
            
        \hline
        \dots & $i^{12}$ & $i^{11}$ & $i^{10}$ & $i^9$ & $i^8$ & $i^9$ & $i^{10}$ & $i^{11}$ & $i^{12}$ & \dots\\
            
        \hline
        \dots & $i^{11}$ & $i^{10}$ & $i^9$ & $i^8$ & $i^7$ & $i^8$ & $i^9$ & $i^{10}$ & $i^{11}$ & \dots\\
            
        \hline
        \dots & $i^{10}$ & $i^9$ & $i^8$ & $i^7$ & $i^6$ & $i^7$ & $i^8$ & $i^9$ & $i^{10}$ & \dots\\
            
        \hline
        \dots & $i^9$ & $i^8$ & $i^7$ & $i^6$ & $i^5$ & $i^6$ & $i^7$ & $i^8$ & $i^9$ & \dots\\
            
        \hline
        & & & & & & & & & &\\
        \hline
        & & & & & & & & & &\\
        \hline
        & & & & & & & & & &\\
        \hline
        & & & & & & & & & &\\
        \hline
        & & & & & $1$ & & & & &\\
        \hline
    \end{tabular}


\noindent Assuming that I have infinite number of pegs.
\begin{itemize}
    \item The sum of the pegs in the column (A):

\begin{align*}
    \text{Sum} &= i^5+ i^6+ i^7+ i^8+ i^9+ \dots\\
    \Longleftrightarrow
    \sum_{n=5}^{\infty}i^n &= i^5+ \left(i^5\cdot i\right)+ \left(i^5\cdot i^2\right)+ \left(i^5\cdot i^3\right)+ \left(i^5\cdot i^4\right)+ \dots\\
\end{align*}

This is a Geometric Series because it has $a=i^5$ and $r=i$ and follows the general formula of this series: $a+ar+ar^2+ar^3+\dots$. Therefore, this series is finite. Its exact value can be calculated using $\left(\frac{a}{1-r} \right)$ formula:

\begin{align*}
    \sum_{n=5}^{\infty}i^n &= \frac{a}{1-r}\\
    &= \frac{i^5}{1-i}\\
    &= \frac{i^5}{i^2} \quad\left(i+i^2=1\Longleftrightarrow i^2=1-i\right)\\
    &= i^3
\end{align*}

    \item The sum of pegs in the column (B), using a similar method to column (A):

\begin{align*}
    \text{Sum} &= i^6+ i^7+ i^8+ i^9+ i^{10}+ \dots\\
    \Longleftrightarrow
    \sum_{n=6}^{\infty}i^n &= i^6+ \left(i^6\cdot i\right)+ \left(i^7\cdot i^2\right)+ \left(i^6\cdot i^3\right)+ \left(i^6\cdot i^4\right)+ \dots\\
    &= \frac{i^6}{1-i}\\
    &= \frac{i^6}{i^2}\\
    &= i^4
\end{align*}

    
    \item The sum of pegs in column (C) is similar to the first two columns, but starts with $a=i^7$:
    $$\sum_{n=7}^{\infty}i^n = \frac{i^7}{i^2} = i^5$$
    
    \item The column (D) with $a=i^8$:
    $$\sum_{n=8}^{\infty}i^n = \frac{i^8}{i^2} = i^6$$
    
    \item Noticing that as columns are getting further than the column (A), the sum is equals to the sum of the left hand side column multiplying by $i$. Therefore, I have a sequence of sum of each column (from column A to D): $i^3, i^4, i^5, i^6$. \par
    Knowing that, I can predict the sum of next columns: $i^7, i^8, i^9, i^{10},\dots $

    
    
    \item Since the columns are symmetric about the column (A) as (B)=(B'), (C)=(C'), \dots, the total sum of all the pegs equals to sum of column (A) and two times of the rest columns:
    \begin{align*}
        \sum &= i^3+ 2\left(i^4+ i^5+ i^6+ i^7+ i^8+ i^9+ \dots \right)\\
        &= i^3+ 2\left[i^4+ \left(i^4\cdot i\right)+ \left(i^4\cdot i^2\right)+ \left(i^4\cdot i^3\right)+ \left(i^4\cdot i^4\right)+ \left(i^4\cdot i^5\right)+ \dots \right]
        \qquad (1)\\
    \end{align*}
    
    The series $\left(i^4+ \left(i^4\cdot i\right)+ \left(i^4\cdot i^2\right)+ \left(i^4\cdot i^3\right)+ \left(i^4\cdot i^4\right)+ \left(i^4\cdot i^5\right)+ \dots\right) $ is the Geometric Series with $a=i^4$ and $r=i$. The sum of this series:
    
    $$ i^4+ \left(i^4\cdot i\right)+ \left(i^4\cdot i^2\right)+ \left(i^4\cdot i^3\right)+ \left(i^4\cdot i^4\right)+ \dots =\frac{i^4}{1-i} =\frac{i^4}{i^2} =i^2 \qquad (2)$$
    
    From (1) and (2), the total sum of all the pegs:
    \begin{align*}
        \sum &= i^3+ 2\left[i^4+ \left(i^4\cdot i\right)+ \left(i^4\cdot i^2\right)+ \left(i^4\cdot i^3\right)+ \left(i^4\cdot i^4\right)+ \left(i^4\cdot i^5\right)+ \dots \right] \\
        &= i^3+ 2\cdot i^2\\
        &= \left(i^3+ i^2\right) + i^2\\
        &= i+ i^2\\
        &= 1
    \end{align*}
    
    Therefore, it is possible to move the pegs to the 5th row if I have infinite number of pegs.
    
    
    \item In conclusion: \par
    If I have an infinitive number of pegs, I can move the peg to the 5th row. \par
    However, I just have a finite number of pegs in reality, so I definitely can not move the peg the the 5th row, which means that moving the peg to 4th row is the limit.
    
\end{itemize}    
    
    

    
\end{document}
