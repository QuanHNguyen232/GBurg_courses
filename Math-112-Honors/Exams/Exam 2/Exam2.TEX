\documentclass[12pt]{article}
\usepackage{latexsym}
\usepackage{amsfonts}
\usepackage{amsmath}

%%Formatting
\topmargin -.2in
\textheight 9.2in
\evensidemargin 0in
\oddsidemargin 0in
\textwidth 6.5in
\parskip .1in

\pagestyle{empty}

\begin{document}

\hrule
\vspace{.2cm}

{\Large \noindent Math 112 Honors
\hfill
B\'ela Bajnok}

\vspace{.3cm}
\hrule

{\Large \noindent 
Exam 2
\hfill
Due: 11:59 PM, September 17, 2020}

\vspace{.3cm}
\hrule

\noindent NAME:  Quan Nguyen

\noindent \hrulefill\rule{0pt}{4pt}

\noindent Write and sign the full Honor Pledge here:

\vspace{2mm}

I affirm that I have upheld the highest principles of honesty and integrity in my academic work and have not witnessed a violation of the Honor Code. \par

Quan Nguyen

\vspace{8mm}

\noindent \hrulefill\rule{0pt}{4pt}

\noindent {\bf General instructions -- Please read!}

\begin{itemize}
  
\item The purpose of this exam is to give you an opportunity to explore a complex and challenging question, gain a fuller view of calculus and its applications, and develop some creative writing, problem solving, and research skills.

\item  {\bf All your assertions must be completely and fully justified.} At the same time, you should aim to be as concise as possible; avoid overly lengthy arguments and unnecessary components.  Your grade will be based on both mathematical accuracy and clarity of presentation.

\item 
Your finished exam should read as an article, consisting of complete sentences, thorough explanations, and exhibit correct grammar and punctuation.

\item I encourage you to prepare your exam using LaTeX.  However, you may use instead other typesetting programs that you like, and you may use hand-writing or hand-drawing for some parts of your exam.  In any case, {\bf the final version that you submit must be in PDF format.}  

\item It is acceptable (and even encouraged) to discuss the exams with other students in your class or with the PLA.  However,  {\bf  you must individually write up all parts of your exams.}

\item {\bf You may use the text, your notes, and your homework, but no other sources.}
 
\item You must write out a complete, honest, and detailed acknowledgment of all assistance you received and all resources you used (including other people) on all written work submitted for a grade.

\item Submit your exam to me by email at bbajnok@gettysburg.edu.  {\bf Your completed exam is due not later than 11:59 PM, September 17, 2020.}







\end{itemize}

\noindent {\bf Good luck!}


\newpage

\begin{center}

{\Large Sum(o) Wrestling} 

\end{center}

\vspace*{.2in}


\noindent In this problem we use Riemann sums to prove that

$$\int_{a}^b x^k \; \mathrm{d}x = \frac{1}{k+1}(b^{k+1}-a^{k+1})$$

holds for $k=1$, $k=2$, and $k=3$.

\begin{enumerate}

\item Use Riemann sums to prove that $$\int_{a}^b x \; \mathrm{d}x = \frac{1}{2}(b^{2}-a^{2}).$$

\item Use Riemann sums to prove that $$\int_{a}^b x^2 \; \mathrm{d}x = \frac{1}{3}(b^{3}-a^{3}).$$

\item Use Riemann sums to prove that $$\int_{a}^b x^3 \; \mathrm{d}x = \frac{1}{4}(b^{4}-a^{4}).$$

\end{enumerate}

\vspace*{.2in}

\newpage

\begin{center}
\underline {\Large {My work} }
\end{center}


\begin{enumerate}
    \item Use Riemann sums to prove that $$\int_{a}^b x \; \mathrm{d}x = \frac{1}{2}(b^{2}-a^{2}).$$\par
    The left part: $\int_{a}^b x \; \mathrm{d}x$ is equal to the Riemann Sum of $x$ from $a$ to $b$. \par
    The interval $\left[a, b\right]$ is divided into $n$ sub-intervals equally, so each sub-interval has a length: $$\Delta x = \frac{b-a}{n}$$
    
    The right endpoint location $x$ of each sub-interval will be:
    \begin{align*}
        & \left(x_1, x_2,\dots, x_{n-2}, x_{n-1}, x_n\right)\\
        =& \left(a+\Delta x, a+2\Delta x,\dots, a+\left(n-2\right)\Delta x, a+\left(n-1\right)\Delta x, b \right)
    \end{align*}

    The right endpoint Riemann Sum:
    \begin{align*}
    R_n
    &= \Delta x\left[f\left(a+\Delta x\right)+ f\left(a+2\Delta x\right)
    +\dots+ f\left(a+\left(n-1\right)\Delta x\right)+b \right]\\
    &= \Delta x\left[f\left(x_1\right)+ f\left(x_2\right)+ f\left(x_3\right)+\dots f\left(x_{n-1}\right)+ f\left(x_n\right) \right] \\
    &=\Delta x\cdot\sum_{i=1}^{n}f\left(x_i\right)\\
    &=\frac{b-a}{n}\cdot\sum_{i=1}^{n}\left(a +\frac{b-a}{n}\cdot i \right)\\
    &=\frac{b-a}{n}\cdot\left(a\sum_{i=1}^{n}1+\frac{b-a}{n}\sum_{i=1}^{n}i\right)\\
    &=\frac{b-a}{n}\cdot\left[a \cdot n+ \frac{b-a}{n}\cdot \frac{n\left(n+1\right)}{2} \right]\\
    &=\frac{b-a}{n}\cdot a\cdot n+ \frac{\left(b-a\right)^2}{n^2}\cdot \frac{n\left(n+1\right)}{2} \\
    &=a\left(b-a \right)+ \frac{n\left(b-a\right)^2+\left(b-a\right)^2}{2n} \\
\end{align*}
\begin{align*}
    \lim_{n\to\infty}R_n &=\lim_{n\to\infty}\left[a\left(b-a \right)+ \frac{n\left(b-a\right)^2+\left(b-a\right)^2}{2n}\right]\\
    &=\lim_{n\to\infty} \Bigg[ a\left(b-a \right)+ \frac{n \left[ \left( b-a\right)^2+ \frac{1}{n} \left(b-a\right)^2 \right]}{2n} \Bigg]\\
    &=\lim_{n\to\infty} a\left(b-a \right)+ \lim_{n\to\infty}\left[ \frac{ \left( b-a\right)^2+ \frac{1}{n} \left(b-a\right)^2}{2}\right] \\
    &=a\left(b-a \right)+ \frac{ \left( b-a\right)^2}{2} \\
    &=\frac{2a\left(b-a \right)}{2}+ \frac{ \left( b-a\right)^2}{2} \\
    &=\frac{\left(2ab-2a^2 \right)+ \left(b^2-2ab+a^2 \right)}{2} \\
    &=\frac{1}{2} \left(b^2-a^2 \right) \\
\end{align*} \par
    Thus, the equation $\int_{a}^b x \; \mathrm{d}x= \frac{1}{2} \left(b^2-a^2 \right)$ is correct.\paragraph{}
    
    \newpage
    
    \item Use Riemann sums to prove that $$\int_{a}^b x^2 \; \mathrm{d}x = \frac{1}{3}(b^{3}-a^{3}).$$ \par
    Similarly to question 1: \par
    \begin{align*}
    R_n &=\Delta x\cdot\sum_{i=1}^{n}f\left(x_i\right)^2\\
    &=\frac{b-a}{n}\cdot\sum_{i=1}^{n}\left(a +\frac{b-a}{n}\cdot i \right)^2\\
    &=\frac{b-a}{n}\cdot\sum_{i=1}^{n}\left(a^2+ 2a\cdot \frac{b-a}{n} \cdot i +\frac{\left( b-a\right)^2}{n^2}\cdot i^2 \right)\\
    &=\frac{b-a}{n}\left(a^2\cdot \sum_{i=1}^{n}1+ 2a\cdot \frac{b-a}{n}\cdot \sum_{i=1}^{n}i +\frac{\left( b-a\right)^2}{n^2}\cdot \sum_{i=1}^{n}i^2 \right)\\
    &=\frac{b-a}{n}\left(a^2\cdot n+ 2a\cdot \frac{b-a}{n}\cdot \frac{n\left( n+1\right)}{2} +\frac{\left( b-a\right)^2}{n^2}\cdot \frac{n\left( n+1\right) \left( 2n+1\right)}{6} \right)\\
    &=a^2\left( b-a\right)
    + \frac{n\cdot2a\left( b-a\right)^2+2a\left(b-a \right)^2}{2n}
    + \frac{2n^2\left(b-a\right)^3+ 3n\left(b-a\right)^3+ \left(b-a\right)^3}{6n^2}\\
    &=a^2\left( b-a\right)
    + a\left(b-a\right)^2\left(1+ \frac{1}{n}\right)
    + \frac{\left(b-a\right)^3 \left(2+\frac{3}{n}+ \frac{1}{n^3} \right)}{6}
\end{align*}


\begin{align*}
    \lim_{n\to\infty}R_n 
    &=\lim_{n\to\infty} a^2\left( b-a\right)
    + \lim_{n\to\infty}\left[ a\left(b-a\right)^2 \left(1+ \frac{1}{n} \right)\right]
    + \lim_{n\to\infty}\left[ \frac{\left(b-a\right)^3 \left(2+\frac{3}{n}+ \frac{1}{n^3} \right)}{6} \right]\\
    &=\left(a^2b-a^3\right)
    + \left(ab^2-2a^2b+a^3\right)
    + \frac{1}{3}\left(b^3-3ab^2+3a^2b-a^3\right)\\
    &= \frac{1}{3} \left(3a^2b-3a^3+ 3ab^2-6a^2b+3a^3+ b^3-3ab^2+3a^2b-a^3 \right)\\
    &= \frac{1}{3}\left(b^3-a^3\right)
\end{align*}
    Thus, the equation $\int_{a}^b x^2 \; \mathrm{d}x= \frac{1}{3} \left(b^3-a^3 \right)$ is correct.\paragraph{}
    
    \newpage

    \item Use Riemann sums to prove that $$\int_{a}^b x^3 \; \mathrm{d}x = \frac{1}{4}(b^{4}-a^{4}).$$ \par
    Similarly to question 1: \par
    \begin{align*}
    \begin{split}
    R_n &=\Delta x\cdot\sum_{i=1}^{n}f\left(x_i\right)^3\\
    &=\frac{b-a}{n}\cdot\sum_{i=1}^{n}\left(a +\frac{b-a}{n}\cdot i \right)^3\\
    &=\frac{b-a}{n}\cdot\sum_{i=1}^{n} \left[a^3+ 3a^2\cdot i\cdot\frac{b-a}{n}+ 3a\cdot i^2\cdot\frac{\left(b-a\right)^2}{n^2} + i^3\cdot\frac{\left(b-a\right)^3}{n^3}\right]\\
    &=\frac{b-a}{n}\left[ a^3\sum_{i=1}^{n}1 
    + \frac{3a^2\left(b-a\right)}{n} \sum_{i=1}^{n}i
    + \frac{3a\left(b-a\right)^2}{n^2} \sum_{i=1}^{n}i^2
    + \frac{\left(b-a\right)^3}{n^3} \sum_{i=1}^{n}i^3 \right]\\
    &=\frac{a^3\left(b-a\right)}{n}\cdot n
    + \frac{3a^2\left(b-a\right)^2}{n^2} \cdot\frac{n\left(n+1\right)}{2}
    + \frac{3a\left(b-a\right)^3}{n^3} \cdot\frac{n\left(n+1\right)\left(2n+1\right)}{6}\\
    &\qquad\qquad\qquad\qquad\qquad\qquad\qquad\qquad\qquad\qquad
    \qquad\quad
    + \frac{\left(b-a\right)^4}{n^4} \cdot \frac{n^2\left(n+1 \right)^2}{4}\\
    &=a^3\left(b-a\right)
    + \frac{3}{2} a^2\left(b-a\right)^2 \left(1+\frac{1}{n}\right)
    + \frac{1}{2} a\left(b-a\right)^3 \left(2+ \frac{3}{n}+ \frac{1}{n^2} \right) \\
    &\qquad\qquad\qquad\qquad\qquad\qquad\qquad\qquad
    \qquad\qquad\qquad
    + \frac{1}{4} \left(b-a\right)^4 \left(1+ \frac{2}{n}+ \frac{1}{n^2} \right)
    \end{split}
    \end{align*}

\begin{align*}
    \lim_{n\to\infty}R_n &=a^3\left(b-a\right)
    + \frac{3}{2} a^2\left(b-a\right)^2 \lim_{n\to\infty}\left(1+\frac{1}{n}\right)
    + \frac{1}{2} a\left(b-a\right)^3 \lim_{n\to\infty}\left(2+ \frac{3}{n}+ \frac{1}{n^2} \right)\\
    &\qquad\qquad\qquad\qquad\qquad\qquad\qquad\qquad
    \qquad\quad
    +\frac{1}{4} \left(b-a\right)^4 \lim_{n\to\infty} \left(1+ \frac{2}{n}+ \frac{1}{n^2} \right)\\
    &=\left(a^3b-a^4\right) 
    + \frac{3}{2} \left(a^2b^2-2a^3b+a^4\right)
    + \left(ab^3-3a^2b^2+3a^3b-a^4\right) \\
    &\qquad\qquad\qquad\qquad\qquad\qquad\qquad\qquad
    \qquad
    + \frac{1}{4} \left(b^4-4ab^3+6a^2b^2-4a^3b+a^4 \right) \\
    &= \frac{1}{4} \left(4a^3b-4a^4\right)
    + \frac{1}{4} \left(6a^2b^2-12a^3b+6a^4\right)\\
    &\qquad\quad
    + \frac{1}{4} \left(4ab^3-12a^2b^2+12a^3b-4a^4\right)
    + \frac{1}{4} \left(b^4-4ab^3+6a^2b^2-4a^3b+a^4 \right) \\
    %+ \frac{3}{2} a^2\left(b-a\right)^2
    %+ a\left(b-a\right)^3
    %+ \frac{1}{4} \left(b-a\right)^4 \\
    %&= \frac{4a^3b-4a^4}{4}
    %+ \frac{6a^2b^2-12a^3b+6a^4}{4}
    %+ \frac{4ab^3-12a^2b^2+12a^3b-4a^4}{4}\\&
    %+ \frac{b^4-4ab^3+6a^2b^2-4a^3b+a^4}{4}\\
    &=\frac{1}{4} \left(b^4-a^4 \right)
\end{align*}
    Thus, the equation $\int_{a}^b x^3 \; \mathrm{d}x= \frac{1}{4} \left(b^4-a^4 \right)$ is correct.\paragraph{}
\end{enumerate}

\newpage

\text{In summary} \par
With $k=1: \int_{a}^b x \; \mathrm{d}x = \frac{1}{2}(b^{2}-a^{2}).
\Longleftrightarrow \int_{a}^b x^1 \; \mathrm{d}x = \frac{1}{1+1}(b^{1+1}-a^{1+1}).$ \par
With $k=2: \int_{a}^b x^2 \; \mathrm{d}x = \frac{1}{3}(b^{3}-a^{3}).
\Longleftrightarrow \int_{a}^b x^2 \; \mathrm{d}x = \frac{1}{2+1}(b^{2+1}-a^{2+1}).$ \par
With $k=3: \int_{a}^b x^3 \; \mathrm{d}x = \frac{1}{4}(b^{4}-a^{4}).
\Longleftrightarrow \int_{a}^b x^3 \; \mathrm{d}x = \frac{1}{3+1}(b^{3+1}-a^{3+1}).$ \par
It is obvious that for every integral of $x^k$ from $a$ to $b$, the result is always equal to: $$\frac{1}{k+1} \left(b^{k+1}-a^{k+1} \right)$$. \par

Thus, the equation $\int_{a}^b x^k \; \mathrm{d}x= \frac{1}{k+1} \left(b^{k+1}-a^{k+1} \right)$ is correct.






\end{document}
