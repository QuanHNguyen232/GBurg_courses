\documentclass[12pt]{article}
\usepackage{latexsym}
\usepackage{amsfonts}
\usepackage{amsmath}
\usepackage{xspace}
\usepackage{tikz}
\usepackage{tikz-3dplot}
\usepackage{graphicx}
\usepackage{pst-solides3d}
\usepackage[utf8]{inputenc}
\usetikzlibrary{calc}
\usepackage{amssymb}
\usepackage{natbib}


%%Formatting
\topmargin -.2in
\textheight 9.2in
\evensidemargin 0in
\oddsidemargin 0in
\textwidth 6.5in
\parskip .1in

\pagestyle{empty}

\begin{document}

\hrule
\vspace{.2cm}

{\Large \noindent Math 112 Honors
\hfill
B\'ela Bajnok}

\vspace{.3cm}
\hrule

{\Large \noindent 
Exam 3
\hfill
Fall 2020}

\vspace{.3cm}
\hrule

\noindent NAME:  Quan Nguyen

\noindent \hrulefill\rule{0pt}{4pt}

\noindent Write and sign the full Honor Pledge here:

\vspace{2mm}

I affirm that I have upheld the highest principles of honesty and integrity in my academic work and have not witnessed a violation of the Honor Code. \par

Quan Nguyen

\vspace{8mm}

\noindent \hrulefill\rule{0pt}{4pt}

\noindent {\bf General instructions -- Please read!}

\begin{itemize}
  
\item The purpose of this exam is to give you an opportunity to explore a complex and challenging question, gain a fuller view of calculus and its applications, and develop some creative writing, problem solving, and research skills.

\item  {\bf All your assertions must be completely and fully justified.} At the same time, you should aim to be as concise as possible; avoid overly lengthy arguments and unnecessary components.  Your grade will be based on both mathematical accuracy and clarity of presentation.

\item 
Your finished exam should read as an article, consisting of complete sentences, thorough explanations, and exhibit correct grammar and punctuation.

\item I encourage you to prepare your exam using LaTeX.  However, you may use instead other typesetting programs that you like, and you may use hand-writing or hand-drawing for some parts of your exam.  In any case, {\bf the final version that you submit must be in PDF format.}  

\item It is acceptable (and even encouraged) to discuss the exams with other students in your class or with the PLA.  However,  {\bf  you must individually write up all parts of your exams.}

\item {\bf You may use the text, your notes, and your homework, but no other sources.}
 
\item You must write out a complete, honest, and detailed acknowledgment of all assistance you received and all resources you used (including other people) on all written work submitted for a grade.

\item Submit your exam to me by email at bbajnok@gettysburg.edu by the deadline announced in class.







\end{itemize}

\noindent {\bf Good luck!}


\newpage

\begin{center}

{\Large Different Viewpoints} 

\end{center}

\vspace*{.2in}


\noindent  
A three-dimensional object is given with the following views: its front view is a square, its side view is a triangle, and its top view is a circle.

1.  Describe the object precisely in terms of its cross sections with respect to a particular direction.  (This is a standard question during job interviews at certain companies such as Microsoft.)  Note that there are two different objects with the given views---describe both.

2.  Find the volume of both objects in terms of the radius of the top view circle.


\newpage

\begin{center}
\underline {\Large {My work} }
\end{center}

\section*{Case 1:}
    
    \begin{enumerate}
        \item Describe \par
    
    Viewing from the top, the object has circle shape with $r$ as its radius.
    
    From the front, that object looks like a square whose side lengths are the same as the diameter of the circle: $2r$, so the size of that square is $2r$ x $2r$.
    
    The cross section of this object is a equilateral triangle (side view on $xz$ plane) because its height $OS$ is a midperpendicular line to the base with $O$ is the center of the based circle. As the cross section moves away from the origin and along the $y$-axis, its height remains the same as the square's height ($OS=2r$), but its base becomes smaller depending on the circle.
    \par
    

    %------------------------------------------
    \tdplotsetmaincoords{70}{110}
    \begin{tikzpicture}[tdplot_main_coords, scale =1]
        \draw[thick,->] (-5,0,0) -- (5,0,0) node[below left]{$x$};
        \draw[thick,->] (0,-5,0) -- (0,5,0) node[below left]{$y$};
        \draw[thick,->] (0,0,-2) -- (0,0,7) node[below left]{$z$};
        \draw (0,0,0) node[below left, color=gray]{$O$};
        \draw (0,3,0) node[below right]{$r$};
        \draw (0,-3,0) node[above left]{$-r$};
        \draw (0,0,6) node[above right]{S$(0,0,2r)$};
        \draw (3,0,0) node[below right]{$r$};
        
        % The base
	    \draw[dashed,color=gray] (0,-3,0) arc (270:90:3 and 3);
	    \draw[semithick] (0,-3,0) arc (-90:90:3 and 3);
	    
	    % The square
	    \draw[semithick] (0,-3,0) -- (0,-3,6);
	    \draw[semithick] (0,3,0) -- (0,3,6);
	    \draw[semithick] (0,-3,6) -- (0,3,6);
	    
	    % Triangles
	    \filldraw[draw=black, fill=gray, opacity=0.2] (3,0,0) -- (-3,0,0) -- (0,0,6);
	    \filldraw[draw=black, fill=gray, opacity=0.6] (2.5981,1.5,0) -- (-2.5981,1.5,0) -- (0,1.5,6);
	    \draw[semithick] (0,1.5,0) -- (0,1.5,6);
	    \draw[draw=gray, thick] (2.5981,1.5,0) -- (0,1.5,0);
	    \draw[dashed, draw=gray, thin] (2.5981,1.5,0) -- (2.5981,0,0);
	    \draw (0,1.5,0) node[above right]{B($0,y,0$)};
	    \draw (2.5981,1.5,0) node[below right]{A($x,y,0$)};
	    \draw (-2.5981,1.5,0) node[above right]{C($-x,y,0$)};
	    
        
        %--------DRAFT---------
        %\draw[dashed,color=gray] (0,0) arc (-90:90:0.5 and 1.5);% right half of the left ellipse
        %\draw[semithick] (0,0,0) arc (270:90:0.5 and 1.5);% left half of the left ellipse
    	%\draw[semithick] (0,0,0) -- (0,4,1);% bottom line
	    %\draw[semithick] (0,0,3) -- (0,4,2);% top line
	    %\draw[semithick] (0,4,1.5) ellipse (0.166 and 0.5);% right ellipse
	    %\draw[semithick] (0,0,0) ellipse (2.5 and 4);
	    %\draw (-1,1.5) node {$\varnothing d_1$};
	    %\draw (3.3,1.5) node {$\varnothing d_2$};
	    %\draw[|-,semithick] (0,-0.5) -- (4,-0.5);
	    %\draw[|->,semithick] (4,-0.5) -- (4.5,-0.5);
	    %\draw (0,-1) node {$x=0$};
	    %\draw (4,-1) node {$x=l$};
        
    \end{tikzpicture}
    %------------------------
    
    \item Volume \par
    \begin{itemize}
    \item The length of cross section's base: \par
    Here is the equation of the circle on $xy$ plane:
    \begin{align*}
        x^2+y^2&=r^2 \\
        \Longleftrightarrow
        x^2&=r^2-y^2 \\
        \Longleftrightarrow
        x&=\pm \sqrt{r^2-y^2}\\
    \end{align*}
    $\Longrightarrow$ $x$ is also the equation to find the length of $AB$. The length of AB must be positive, so the length of AB has the equation: $|x|=\sqrt{r^2-y^2}$.\par
    
    However, the cross section's base (triangle's base) is equal to the length of $AC$. Since the triangle is equilateral, $AC$ is equal to $2AB$. Thus, the length of the triangle's base $AC$ is: $$2|x|=2\sqrt{r^2-y^2}$$
    
    \item Area of the cross section (triangle):
    \begin{align*}
        \text{A}(y) &=\frac{1}{2}\left(\text{height}\cdot\text{base} \right)\\
        &=\frac{1}{2}\left(OS\cdot AC\right)\\
        &=\frac{1}{2}\left(2r\cdot 2\sqrt{r^2-y^2} \right)\\
        &= 2r\sqrt{r^2-y^2}\\
    \end{align*}
    
    \item Volume of the object: \par
    Since the object is symmetric about $z$-axis on $yz$ plane, the object's volume from $-r \to 0$ is equal to object's volume from $0\to r$ on $y$-axis. \par
    \begin{align*}
    \mathrm{V}(y) = \int_{-r}^r \mathrm{A}(y)\; \mathrm{d}y
    &=2 \int_{0}^r \mathrm{A}(y)\; \mathrm{d}y \\
    &=2 \int_{0}^r 2r\sqrt{r^2-y^2} \; \mathrm{d}y\\
    &=4r \int_{0}^r \sqrt{r^2-y^2} \; \mathrm{d}y
    \end{align*}
    
    I let $y=r\sin{t}$ because when squaring both sides of the equation, $``y"\to ``y^2"$ and $``r\sin{t}"\to ``r^2\sin^2{t}"$. Then leave $r^2$ as factor, $1-\sin^2{t}=\cos^2{t}$ which cancels the square root: \par
    \begin{equation*}
        y=r\sin{t}\\
        \Longleftrightarrow
        \begin{cases}
            \mathrm{d}y = r \cos{t}\; \mathrm{d}t\\
            t=\arcsin{\frac{y}{r}}
        \end{cases}
    \end{equation*}
    
    
    \begin{align*}
    \mathrm{V}(y) &= 4r \int_{0}^r \sqrt{r^2-y^2} \; \mathrm{d}y \\
    \Longrightarrow \mathrm{V}(y)&= 4r \int_{0}^r r\cos{t}\cdot\sqrt{r^2-r^2\sin^2{t}} \; \mathrm{d}t \\
    &= 4r \int_{0}^r r\cos{t}\cdot \sqrt{r^2 \left(1-\sin^2{t}\right)} \; \mathrm{d}t \\
    &= 4r \int_{0}^r r\cos{t}\cdot\sqrt{r^2 \cos^2{t}} \; \mathrm{d}t \\
    &= 4r \int_{0}^r r\cos{t}\cdot r \cos{t} \; \mathrm{d}t \\
    &= 4r \int_{0}^r r^2 \cos^2{t}\; \mathrm{d}t \\
    &= 4r^3 \int_{0}^r \cos^2{t}\; \mathrm{d}t \\
    &= 4r^3 \int_{0}^r \frac{1+\cos{2t}}{2}\; \mathrm{d}t \\
    &= 2r^3\int_{0}^r 1\; \mathrm{d}t +2r^3 \int_{0}^r \cos{2t}\; \mathrm{d}t \\
    &= 2r^3\int_{0}^r 1\; \mathrm{d}t
    +r^3 \int_{0}^r 2\cos{2t}\; \mathrm{d}t \\
    &= \left(2r^3 t \right)_0^r
    +\left(r^3 \sin{2t} \right)_0^r \\
    &= \left(2r^3 \arcsin{\frac{y}{r}}
    \right)_0^r
    +\left[r^3 \sin{\left( 2\arcsin{\frac{y}{r}} \right)} \right]_0^r \\
    &= \left(2r^3\cdot \frac{\pi}{2}-0 \right)
    +\left(0-0 \right) \\
    &= \pi r^3 \\
    \end{align*}
    
    In summary, the volume of the object that has triangle cross sections is $\pi r^3$.
    
    \end{itemize}

    
    \end{enumerate}

    
    \vspace{2cm}
\newpage
\section*{Case 2:}
    
    \begin{enumerate}
        \item Describe \par
        Viewing from the top, the object has circle shape with $r$ as its radius, which is the same as Case 1.
        
        However, from the front in Case 2, the object now is a equilateral triangle. The triangle has base length of $2r$ (equal to diameter of the based circle) and height of $2r$ so that it can create a $2r$ x $2r$ square when viewing from the side. The triangle is equilateral because its height is also $Oz$-axis, which is midperpendicular to the base.
        
        The cross section of this object is a square (side view) with the size of $2r$ x $2r$ on the $xz$ plane as I explained above. That cross section when moving away from origin $O$, along the $y$-axis, it loses square shape because its height decrease steadily due to triangle's side while its base decrease exponentially due to the circle.
        \par
    
    %--------------------------
    \tdplotsetmaincoords{70}{110}
    \begin{tikzpicture}[tdplot_main_coords, scale =1]
    \draw[thick,->] (-5,0,0) -- (5,0,0) node[below left]{$x$};
    \draw[thick,->] (0,-5,0) -- (0,5,0) node[below left]{$y$};
    \draw[thick,->] (0,0,-2) -- (0,0,7) node[below left]{$z$};
        
    % The base
	\draw[dashed,color=gray] (0,-3,0) arc (270:90:3 and 3);
	\draw[semithick] (0,-3,0) arc (-90:90:3 and 3);
    \draw (0,0,0) node[below left]{$O$};
    \draw (0,3,0) node[above right]{B($0,r,0$)};
    \draw (0,-3,0) node[above left]{$-r$};
    \draw (0,0,6) node[left]{A($0,0,2r$)};
	
	% The triangle
	\draw[semithick] (0,-3,0) -- (0,0,6);
	\draw[semithick] (0,3,0) -- (0,0,6);

	% Squares
	\filldraw[draw=black, fill=gray, opacity=0.2] (3,0,0) -- (-3,0,0) -- (-3,0,6) -- (3,0,6);
	\filldraw[draw=black, fill=gray, opacity=0.6] (2.5981,1.5,0) -- (-2.5981,1.5,0) -- (-2.5981,1.5,3) -- (2.5981,1.5,3);
	\draw[semithick] (-2.5981,1.5,0) -- (2.5981,1.5,0);
	\draw[semithick] (0,1.5,0) -- (0,1.5,3);

    \end{tikzpicture}
    %----------------------------
    
    
    \item Volume \par
    \begin{itemize}
    \item Height of the cross section:\par
    
    Because the triangle (on $yz$ plane) is symmetric about $z$-axis, the change in cross section's height from $O\to -r$ is the same as that from $O\to r$, so I only need to find the equation of the line $AB$ (from $O\to r$).
    
    The general formula of linear line $AB$: $$z = my + b$$ \par
    The $z$ intercept of the line $AB$ on $yz$ plane is $A(0,0,2r)$, and the $y$ intercept is $B(0,r,0)$. Plug these two coordinations into the general formula, I have a system of two equations:
    \begin{equation*}
    \begin{cases}
        2r=0 + b\\
        0=rm+b
    \end{cases} \\
    \Longleftrightarrow
    \begin{cases}
        b=2r\\
        m=-2
    \end{cases}
    \end{equation*}
    Therefore, the equation for height of square is:
    $$z=-2y+2r$$
    
    \par
    \item Equation for base of the cross section (on $xy$ plane), same as in Case 1:
    $$2x=2\sqrt{r^2-y^2}$$ \par
    \item Area of cross section (rectangle): 
    \begin{align*}
        \mathrm{A}(y)&=\text{height} \cdot \text{base}\\
        &=z\cdot 2x \\
        &=\left(-2y+2r\right)\cdot 2\sqrt{r^2-y^2}\\
        &= 4\left(r-y\right)\sqrt{r^2-y^2}\\
        &= 4r\sqrt{r^2-y^2}-4y\sqrt{r^2-y^2}
    \end{align*}
    
    \item Volume of the object (similar to Case 1):. \par
    \begin{align*}
    \mathrm{V}(y) = \int_{-r}^r \mathrm{A}(y)\; \mathrm{d}y
    &=2 \int_{0}^r \mathrm{A}(y)\; \mathrm{d}y \\
    &=2 \int_{0}^r\left( 4r\sqrt{r^2-y^2}-4y\sqrt{r^2-y^2}\right) \; \mathrm{d}y\\
    &=2\cdot 4r \int_{0}^r \sqrt{r^2-y^2}\; \mathrm{d}y
    - 8\int_{0}^r y\sqrt{r^2-y^2} \; \mathrm{d}y\\
    &=2\pi r^3
    - 8\int_{0}^r y\sqrt{r^2-y^2} \; \mathrm{d}y
    \quad \left(4r \int_{0}^r \sqrt{r^2-y^2} \; \mathrm{d}y=\pi r^3\right)
    \end{align*}
    
    Assuming that:
    \begin{align*}
        &t=\sqrt{r^2-y^2}\\
        \Longrightarrow &t^2 = r^2-y^2\\
        \Longrightarrow &2t\; \mathrm{d}t = -2y\;\mathrm{d}y\\
        \Longrightarrow &t\; \mathrm{d}t = -y\;\mathrm{d}y\\
    \end{align*}
    
    \begin{align*}
        \mathrm{V}(y) &=2\pi r^3
        - 8\int_{0}^r y\sqrt{r^2-y^2} \; \mathrm{d}y\\
        \Longrightarrow \mathrm{V}(y)
        &=2\pi r^3
        + 8\int_{0}^r t^2 \; \mathrm{d}t\\
        &=2\pi r^3
        + 8\left(\frac{t^3}{3}\right)_0^r\\
        &=2\pi r^3
        + \frac{8}{3}\left[\left(r^2-y^2 \right) \sqrt{r^2-y^2}\right]_0^r \\
        &=2\pi r^3
        + \frac{8}{3}\left(0 -r^2 \sqrt{r^2} \right) \\
        &=2\pi r^3- \frac{8}{3}r^3\\
        &=2r^3 \left(\pi -\frac{4}{3}\right)
    \end{align*}

        \end{itemize}
    
    In summary, the volume of the object that has rectangle cross sections is $ 2r^3\left(\pi -\frac{4}{3}\right)$.
    
    \end{enumerate}

    



\end{document}
