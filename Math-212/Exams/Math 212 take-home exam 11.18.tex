\documentclass[10pt]{article}
\pagestyle{plain} \setlength{\textwidth}{12cm}
\setlength{\oddsidemargin}{2cm} \setlength{\evensidemargin}{2cm}
\setlength{\textheight}{23cm}
\usepackage{amssymb,latexsym,amsmath,amsthm,verbatim,calc}
\usepackage{graphicx,epsfig,epstopdf,amssymb,color,makeidx,float,caption,sidecap}

\begin{document}
\title{Math 212 take-home exam}
\author{Quan Nguyen}
\maketitle

\bigskip
\noindent
{\bf Honor Pledge:}
\par
\noindent I affirm that I have upheld the highest principles of honesty and integrity in my academic work and have not witnessed a violation of the Honor Code.
\par
\bigskip
\noindent Quan Nguyen

\bigskip
\noindent

\section*{Question 1}
$$ \begin{bmatrix} -3 & k \\ -1 & -1 \end{bmatrix}. $$
\subsection*{(a)}
\noindent The characteristic polynomial is:
\begin{align*}
    det(A -\lambda I) &= det
    \begin{bmatrix}
        -3-\lambda & k \\
        -1 & -1-\lambda
    \end{bmatrix} \\
    &= (-3-\lambda)(-1-\lambda) + k \\
    &= \lambda^2 + 4\lambda + 3 + k.
\end{align*}
\noindent Matrix $A$ has only one eigenvalue of multiplicity of 2 if $\lambda^2 + 4\lambda + 3 + k = 0$ and $\lambda$ has one value. Thus $k$ must equal to 1:
\begin{align*}
    \lambda^2 + 4\lambda + 3 + k &= 0 \\
    \lambda^2 + 4\lambda + 4 &= 0 \\
    (\lambda + 2)^2 &= 0.
\end{align*}
\noindent Therefore, when $k=1$, matrix $A$ has single eigenvalue -2 with multiplicity of 2.


\subsection*{(b)}
\noindent With $k=1$, the matrix $A$ will be:
\begin{equation*}
    A = \begin{bmatrix}
        -3 & 1 \\
        -1 & -1
    \end{bmatrix}.
\end{equation*}

\noindent From (a), we know that the only eigenvalue of $A$ is -2. Let $\Vec{v_1}$ be the eigenvector corresponding for that eigenvalue:
\begin{align*}
    A\Vec{v_1} &= -2\Vec{v_1} \\
    (A+2I)\Vec{v_1} &= \Vec{0} \\
    \begin{bmatrix}
        -1 & 1 \\
        -1 & 1
    \end{bmatrix} \Vec{v_1} &= \Vec{0}.
\end{align*}

\noindent Let $a, b$ be entries of $\Vec{v_1}$. The eigenvector equations are:
\begin{align*}
    -a + b &= 0 \\
    -a + b &= 0.
\end{align*}
\noindent Both of these equations yield $a=b$, so eigenvectors corresponding to -2 can be a scalar multiple of:
\begin{equation*}
    \begin{bmatrix} 1 \\ 1 \end{bmatrix}.
\end{equation*}

\noindent We have $\Vec{v_1} = \begin{bmatrix} 1 \\ 1 \end{bmatrix}$ is the only eigenvector corresponding to eigenvalue -2, so there is no basis of eigenvectors for matrix $A$ when $k=1$. \par

\noindent To form a Jordan basis, we need to find another vector. Let's call it $\Vec{v_2}$. If we choose any vector $\Vec{w}$ independent from $\Vec{v_1}$, knowing that $(A-\lambda I)\Vec{w}$  will be a scalar multiple of $\Vec{v_1}$, then we can get $\Vec{v_2}$ by scaling appropriately $\Vec{w}$. \par
\noindent Let's take:
\begin{equation*}
    \Vec{w} = \begin{bmatrix} 0 \\ 1 \end{bmatrix}.
\end{equation*}
\noindent We must show that $(A-\lambda I)\Vec{w} = \Vec{v_1}$. Then we compute:
\begin{align*}
    (A+2I)\Vec{w} &=
    \begin{bmatrix}
        -1 & 1 \\
        -1 & 1
    \end{bmatrix}
    \begin{bmatrix}
        0 \\ 1
    \end{bmatrix} \\
    &= \begin{bmatrix}
        1 \\ 1
    \end{bmatrix} \\
    &= \Vec{v_1}.
\end{align*}

\noindent So we take $\Vec{v_2} = \Vec{w}$. Therefore, the Jordan basis will be:
\begin{equation*}
    \mathcal{B} = \left\{
    \begin{bmatrix}
        1 \\ 1
    \end{bmatrix},
    \begin{bmatrix}
        0 \\ 1
    \end{bmatrix}
    \right\}.
\end{equation*}



\section*{Question 2}
$$A = 
\begin{bmatrix}
    \frac{3}{4} & \frac{1}{4} \\
    \frac{1}{4} & \frac{3}{4}
\end{bmatrix}.$$

\subsection*{(a)}
\noindent The characteristic polynomial is:
\begin{equation*}
    det(A-\lambda I) = \left(\frac{3}{4} -\lambda \right)^2 - \frac{1}{16}.
\end{equation*}

\noindent The eigenvalues of $A$ are:
\begin{align*}
    det(A-\lambda I) = \left(\frac{3}{4} -\lambda \right)^2 - \frac{1}{16} &= 0 \\
    \left(\frac{3}{4} -\lambda \right)^2 &= \frac{1}{16} \\
    (2\lambda -1)(\lambda -1) &= 0.
\end{align*}

\noindent So we have two distinct eigenvalues $\frac{1}{2}$ and 1.

\noindent For eigenvalue $\frac{1}{2}$, the corresponding eigenvector, let it be $\Vec{v_1}$, is:
\begin{align*}
    \left(A- \frac{1}{2}I \right) \Vec{v_1} &= \Vec{0} \\
    \begin{bmatrix}
        \frac{1}{4} & \frac{1}{4} \\
        \frac{1}{4} & \frac{1}{4}
    \end{bmatrix} \Vec{v_1} &= \Vec{0}.
\end{align*}
\noindent Let $a, b$ be entries of $\Vec{v_1}$. The eigenvector equations are:
\begin{align*}
    \frac{1}{4}a + \frac{1}{4}b &= 0 \\
    \frac{1}{4}a + \frac{1}{4}b &= 0.
\end{align*}
\noindent Both of these equations yield $a=-b$, so eigenvectors corresponding to $\frac{1}{2}$ can be a scalar multiple of:
\begin{equation*}
    \Vec{v_1} = \begin{bmatrix} 1 \\ -1 \end{bmatrix}.
\end{equation*}

\noindent For eigenvalue 1, the corresponding eigenvector, let's call it $\Vec{v_2}$, is:
\begin{align*}
    \left(A- I \right) \Vec{v_2} &= \Vec{0} \\
    \begin{bmatrix}
        -\frac{1}{4} & \frac{1}{4} \\
        \frac{1}{4} & -\frac{1}{4}
    \end{bmatrix} \Vec{v_2} &= \Vec{0}.
\end{align*}
\noindent Let $c, d$ be entries of $\Vec{v_2}$. The eigenvector equations are:
\begin{align*}
    -\frac{1}{4}c + \frac{1}{4}d &= 0 \\
    \frac{1}{4}c - \frac{1}{4}d &= 0.
\end{align*}
\noindent Both of these equations yield $c=d$, so eigenvectors corresponding to 1 can be a scalar multiple of:
\begin{equation*}
    \Vec{v_2} = \begin{bmatrix} 1 \\ 1 \end{bmatrix}.
\end{equation*}

\noindent Since $A$ is a 2x2 matrix that has 2 distinct eigenvalues with corresponding eigenvectors, those corresponding eigenvectors form a basis of eigenvectors that span in $\mathbb{R}^2$.

\subsection*{(b)}
\noindent Let $M$ be a 2x2 matrix similar to $A$, so $A = PMP^{-1}$ with columns of $P$ be a basis of eigenvectors, and $M$ be diagonal matrix with entries are eigenvalues since there is a basis of eigenvectors. \par
\begin{equation*}
    P = \begin{bmatrix}
        1 & 1 \\
        -1 & 1
    \end{bmatrix}.
\end{equation*}
and 
\begin{equation*}
    P^{-1} = \frac{1}{2}
    \begin{bmatrix}
        1 & -1 \\
        1 & 1
    \end{bmatrix}.
\end{equation*}

\begin{equation*}
    M = \begin{bmatrix}
        \frac{1}{2} & 0 \\
        0 & 1
    \end{bmatrix}.
\end{equation*}

\noindent Then:
\begin{align*}
    A^k &= PM^kP^{-1} \\
    &= \frac{1}{2}
    \begin{bmatrix}
        1 & 1 \\
        -1 & 1
    \end{bmatrix}
    \begin{bmatrix}
        \frac{1}{2^k} & 0 \\
        0 & 1
    \end{bmatrix}
    \begin{bmatrix}
        1 & -1 \\
        1 & 1
    \end{bmatrix} \\
    &= \frac{1}{2}
    \begin{bmatrix}
        1 & 1 \\
        -1 & 1
    \end{bmatrix}
    \begin{bmatrix}
        \frac{1}{2^k} & \frac{-1}{2^k} \\
        1 & 1
    \end{bmatrix} \\
    &= \frac{1}{2}
    \begin{bmatrix}
        \frac{1}{2^k}+1 & \frac{-1}{2^k}+1 \\
        \frac{-1}{2^k}+1 & \frac{1}{2^k}+1
    \end{bmatrix}.
\end{align*}



\subsection*{(c)}

\begin{equation*}
    A^k\Vec{x} = \frac{1}{2}
    \begin{bmatrix}
        \frac{1}{2^k}+1 & \frac{-1}{2^k}+1 \\
        \frac{-1}{2^k}+1 & \frac{1}{2^k}+1
    \end{bmatrix}
    \begin{bmatrix}
        1 \\ 0
    \end{bmatrix} =
    \frac{1}{2}
    \begin{bmatrix}
        \frac{1}{2^k}+1 \\
        \frac{-1}{2^k}+1
    \end{bmatrix}.
\end{equation*}

\noindent If $k$ is very large, then, $\frac{1}{2^k}$ approaches 0, so we have $A^k\Vec{x}$ with large $k$ will be:
\begin{equation*}
    A^k\Vec{x} = \frac{1}{2}
    \begin{bmatrix}
        1 \\
        1
    \end{bmatrix}.
\end{equation*}



\section*{Question 3}
\noindent Let matrix $A$ be:
$$A = 
\begin{bmatrix}
    0 & \frac{49}{50} & \frac{12}{25} \\
    \frac{1}{2} & 0 & 0 \\
    0 & \frac{1}{2} & 0
\end{bmatrix}.$$

\subsection*{(a)}
\noindent The proportion of juveniles survive each year to become adults the following year is $\frac{1}{2}$.


\subsection*{(b)}

\begin{align*}
    \begin{bmatrix}
        0 & \frac{49}{50} & \frac{12}{25} \\
        \frac{1}{2} & 0 & 0 \\
        0 & \frac{1}{2} & 0
    \end{bmatrix}
    \begin{bmatrix}
        \text{juveniles this year} \\
        \text{adults this year} \\
        \text{seniors this year}
    \end{bmatrix} &= 
    \begin{bmatrix}
        \text{juveniles next year} \\
        \text{adults next year} \\
        \text{seniors next year}
    \end{bmatrix}\\ &=
    \begin{bmatrix}
        \frac{49}{50} \text{ adults this year} + \frac{12}{25} \text{ seniors this year} \\
        \frac{1}{2} \text{ juveniles this year} \\
        \frac{1}{2} \text{ adults this year}
    \end{bmatrix}.
\end{align*}

\noindent Therefore, the total number of offspring seniors has each year, to appear as juveniles next year is $\left(\frac{12}{25}\cdot \text{ number of seniors this year}\right)$. The average number of offspring each senior has will be:
$$\frac{\left(\frac{12}{25}\cdot \text{ number of seniors this year}\right)}{\text{ number of seniors this year}} = \frac{12}{25}.$$




\subsection*{(c)}
\noindent Commands I used in Cocalc:
\begin{itemize}
    \item To create matrix $A$: \par
    A = matrix(QQ, [
    [0, 49/50, 12/25],
    [1/2, 0, 0],
    [0, 1/2, 0] ] ). \par
    To see matrix $A$: show(A).
    \item To find eigenvalues of $A$: show(A.eigenvectors\_right()).
\end{itemize}

\noindent In total, $A$ has 3 eigenvalues which are: $$\frac{4}{5}, -\frac{3}{10}, -\frac{1}{2}.$$



\subsection*{(d)}
\noindent From (c), we have the absolute values of 3 eigenvectors are all less than 1, so in the long run, the population of this organism will approach to 0.


\section*{Question 4}

\noindent Basis of $V$:
$$\mathcal{B} = \left\{ 
\begin{bmatrix} 1 & 0 \\ 0 & 0 \end{bmatrix},
\begin{bmatrix} 0 & 1 \\ 0 & 0 \end{bmatrix},
\begin{bmatrix} 0 & 0 \\ 1 & 0 \end{bmatrix},
\begin{bmatrix} 0 & 0 \\ 0 & 1 \end{bmatrix}
\right\}.$$

\begin{itemize}
    \item
    \begin{align*}
        T\left(\begin{bmatrix} 1 & 0 \\ 0 & 0 \end{bmatrix}\right)
        &= 
        \begin{bmatrix} 1 & 2 \\ 3 & 4 \end{bmatrix}
        \begin{bmatrix} 1 & 0 \\ 0 & 0 \end{bmatrix}
        = \begin{bmatrix} 1 & 0 \\ 3 & 0 \end{bmatrix}.
    \end{align*}
    \noindent The $\mathcal{B}$-coordinate vector is: $\begin{bmatrix} 1 \\ 0 \\ 3 \\ 0 \end{bmatrix}$.
    
    \item
    \begin{align*}
        T\left(\begin{bmatrix} 0 & 1 \\ 0 & 0 \end{bmatrix}\right)
        &= 
        \begin{bmatrix} 1 & 2 \\ 3 & 4 \end{bmatrix}
        \begin{bmatrix} 0 & 1 \\ 0 & 0 \end{bmatrix}
        = \begin{bmatrix} 0 & 1 \\ 0 & 3 \end{bmatrix}.
    \end{align*}
    \noindent The $\mathcal{B}$-coordinate vector is: $\begin{bmatrix} 0 \\ 1 \\ 0 \\ 3 \end{bmatrix}$.
    
    \item
    \begin{align*}
        T\left(\begin{bmatrix} 0 & 0 \\ 1 & 0 \end{bmatrix}\right)
        &= 
        \begin{bmatrix} 1 & 2 \\ 3 & 4 \end{bmatrix}
        \begin{bmatrix} 0 & 0 \\ 1 & 0 \end{bmatrix}
        = \begin{bmatrix} 2 & 0 \\ 4 & 0 \end{bmatrix}.
    \end{align*}
    \noindent The $\mathcal{B}$-coordinate vector is: $\begin{bmatrix} 2 \\ 0 \\ 4 \\ 0 \end{bmatrix}$.
    
    \item
    \begin{align*}
        T\left(\begin{bmatrix} 0 & 0 \\ 0 & 1 \end{bmatrix}\right)
        &= 
        \begin{bmatrix} 1 & 2 \\ 3 & 4 \end{bmatrix}
        \begin{bmatrix} 0 & 0 \\ 0 & 1 \end{bmatrix}
        = \begin{bmatrix} 0 & 2 \\ 0 & 4 \end{bmatrix}.
    \end{align*}
    \noindent The $\mathcal{B}$-coordinate vector is: $\begin{bmatrix} 0 \\ 2 \\ 0 \\ 4 \end{bmatrix}$.
\end{itemize}

\noindent Therefore, the $\mathcal{B}$-matrix for the linear transformation $T$:

\begin{equation*}
    [T]_{\mathcal{B}} =
    \begin{bmatrix}
    1 & 0 & 2 & 0 \\
    0 & 1 & 0 & 2 \\
    3 & 0 & 4 & 0\\
    0 & 3 & 0 & 4
    \end{bmatrix}.
\end{equation*}



\section*{Question 5}
\noindent Let $\Vec{v_1}$ be eigenvector for eigenvalue 1 and $\Vec{v_2}$ be eigenvector for eigenvalue 2. \par
\noindent Since $A$ is a 2x2 matrix with 2 distinct eigenvalue, $\Vec{v_1}, \Vec{v_2}$ form a basis of eigenvectors in $\mathbb{R}^2$. \par
\noindent Let $\Vec{u}$ be any nonzero vector in $\mathbb{R}^2$ and is not itself an eigenvector for eigenvalue 2. Then $\Vec{u}$ can be expressed as:
$$\Vec{u} = c_1\Vec{v_1} + c_2\Vec{v_2} \text{ for some scalars } c_1, c_2 \text{ and } c_1\neq 0. $$

\noindent Note that since $\Vec{v_2}$ be eigenvector corresponding for eigenvalue 2, we have:
\begin{align*}
    A\Vec{v_2} &= 2\Vec{v_2} \text{ then}\\
    (A - 2I) \Vec{v_2} &= \Vec{0}.
\end{align*}

\noindent We want to show that the vector $(A - 2I)\Vec{u}$ is a multiple of $\Vec{v_1}$:

\begin{align*}
    (A - 2I)\Vec{u} &= (A - 2I)(c_1\Vec{v_1} + c_2\Vec{v_2}) \\
    &= c_1A\Vec{v_1} - 2c_1\Vec{v_1} + c_2(A- 2I)\Vec{v_2} \\
    &= c_1\Vec{v_1} - 2c_1\Vec{v_1} \\
    &= c_1\Vec{v_1}(1-2) = -c_1\Vec{v_1}.
\end{align*}

\noindent Therefore, $(A - 2I)\Vec{u}$ is a multiple of $\Vec{v_1}$ which means that $(A - 2I)\Vec{u}$ is an eigenvector for eigenvalue 1.






\end{document} 