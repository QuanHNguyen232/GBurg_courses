\documentclass[10pt]{article}
\pagestyle{plain} \setlength{\textwidth}{12cm}
\setlength{\oddsidemargin}{2cm} \setlength{\evensidemargin}{2cm}
\setlength{\textheight}{23cm}
\usepackage{amssymb,latexsym,amsmath,amsthm,verbatim,calc}
\usepackage{graphicx,epsfig,epstopdf,amssymb,color,makeidx,float,caption,sidecap}

\begin{document}
\title{Math 212 written homework assignment week 6}
\author{Quan Nguyen}
\maketitle

\bigskip
\noindent
{\bf Acknowledgements and sources of help}:
\begin{itemize}
    \item Professor Ben Ken
    \item Tam Nguyen
\end{itemize}

\bigskip
\noindent

\section*{Problem 1}
\begin{equation*}
    A = 
    \begin{bmatrix}
        a & b \\
        c & d
    \end{bmatrix};
    B = 
    \begin{bmatrix}
        0 & 0 \\
        1 & 0
    \end{bmatrix}
\end{equation*}
\noindent First, we need to find the result of $AB$:
\begin{equation*}
    AB = 
    \begin{bmatrix}
        a & b \\
        c & d
    \end{bmatrix}
    \begin{bmatrix}
        0 & 0 \\
        1 & 0
    \end{bmatrix}
    =
    \begin{bmatrix}
        b & 0 \\
        d & 0
    \end{bmatrix}
\end{equation*}
\noindent Then, we find the result of $BA$:
\begin{equation*}
    BA = 
    \begin{bmatrix}
        0 & 0 \\
        1 & 0
    \end{bmatrix}
        \begin{bmatrix}
        a & b \\
        c & d
    \end{bmatrix}
    =
    \begin{bmatrix}
        0 & 0 \\
        a & b
    \end{bmatrix}
\end{equation*}
\noindent To have $AB=BA$, then this must be true:
\begin{equation*}
    \begin{bmatrix}
        b & 0 \\
        d & 0
    \end{bmatrix}
    =
    \begin{bmatrix}
        0 & 0 \\
        a & b
    \end{bmatrix}
\end{equation*}
\noindent Therefore, this must be true the make these two matrices equal to each other:
\begin{itemize}
    \item $b=0$
    \item $d=a$
\end{itemize}
\noindent So, the matrix $A$ must be like this:
\begin{equation*}
    A = 
    \begin{bmatrix}
        a & 0 \\
        c & a
    \end{bmatrix}
    \text{ or }
    \begin{bmatrix}
        d & 0 \\
        c & d
    \end{bmatrix}
    \text{   for all values of $c$}
\end{equation*}

\section*{Problem 2}
\begin{equation*}
    A = 
    \begin{bmatrix}
        a & b \\
        c & d
    \end{bmatrix};
    B = 
    \begin{bmatrix}
        0 & 1 \\
        0 & 0
    \end{bmatrix}
\end{equation*}
\noindent Similar to the problem above, we start with finding the result of $AB$:
\begin{equation*}
    AB = 
    \begin{bmatrix}
        a & b \\
        c & d
    \end{bmatrix}
    \begin{bmatrix}
        0 & 1 \\
        0 & 0
    \end{bmatrix}
    =
    \begin{bmatrix}
        0 & a \\
        0 & c
    \end{bmatrix}
\end{equation*}
\noindent The result of $BA$:
\begin{equation*}
    BA = 
    \begin{bmatrix}
        0 & 1 \\
        0 & 0
    \end{bmatrix}
    \begin{bmatrix}
        a & b \\
        c & d
    \end{bmatrix}
    =
    \begin{bmatrix}
        c & d \\
        0 & 0
    \end{bmatrix}
\end{equation*}
\noindent Then, $AB=BA$ when their results are equal to each other:
\begin{equation*}
    \begin{bmatrix}
        0 & a \\
        0 & c
    \end{bmatrix}
    =
    \begin{bmatrix}
        c & d \\
        0 & 0
    \end{bmatrix}
\end{equation*}
\noindent Therefore, we have these conditions:
\begin{itemize}
    \item $c=0$
    \item $a=d$
\end{itemize}
\noindent So, the matrix $A$ must have this form;
\begin{equation*}
    A = 
    \begin{bmatrix}
        a & b \\
        0 & a
    \end{bmatrix}
    \text{ or }
    \begin{bmatrix}
        d & b \\
        0 & d
    \end{bmatrix}
    \text{  for all values of $b$}
\end{equation*}


\section*{Problem 3}
\noindent I choose problem 2 with $d=1; b=2$ for $A$ and the matrix $b$ for the example:
\begin{equation*}
    A = 
    \begin{bmatrix}
        1 & 2 \\
        0 & 1
    \end{bmatrix};
    B =
    \begin{bmatrix}
        0 & 1 \\
        0 & 0
    \end{bmatrix}
\end{equation*}
\noindent Therefore, we have:
\begin{equation*}
    AB=
    \begin{bmatrix}
        0 & 1 \\
        0 & 0
    \end{bmatrix};
    BA = 
    \begin{bmatrix}
        0 & 1 \\
        0 & 0
    \end{bmatrix}
\end{equation*}
\noindent From the result above, it is obvious that $AB=BA$

\section*{Problem 4}
\begin{equation*}
        A=
        \begin{bmatrix}
            a & b \\
            c & d
        \end{bmatrix}
    \end{equation*}
\noindent My statement: "$AB=BA$ for all 2x2 matrices $B$ if and only if matrix $A$ has $a=d, b=0, c=0$". \par
\begin{itemize}
    \item For "only if" direction: \par
    Assume that $a=d, b=0, c=0$ are true. \par
    Assume matrix $B$ has this general formula:
    \begin{equation*}
        B=
        \begin{bmatrix}
            e & f \\
            g & h
        \end{bmatrix}
    \end{equation*}
    We have:
    \begin{equation*}
        AB=
        \begin{bmatrix}
            a & 0 \\
            0 & a
        \end{bmatrix}
        \begin{bmatrix}
            e & f \\
            g & h
        \end{bmatrix}
        = 
        \begin{bmatrix}
            ae & af \\
            ag & ah
        \end{bmatrix}
    \end{equation*}
    \begin{equation*}
        BA=
        \begin{bmatrix}
            e & f \\
            g & h
        \end{bmatrix}
        \begin{bmatrix}
            a & 0 \\
            0 & a
        \end{bmatrix}
        =
        \begin{bmatrix}
            ae & af \\
            ag & ah
        \end{bmatrix}
    \end{equation*}
    Thus, we can see that $AB=BA$.
    
    \item For "if" direction: \par
    Assume that $AB=BA$ for all matrices $B$. \par
    If we choose a specific matrix $B=
    \begin{bmatrix}
        0 & 0 \\
        1 & 0
    \end{bmatrix}$, then for matrix $A$, $a=d, b=0$ are true. (Problem 1) \par
    Then, if we choose a specific matrix $B=
    \begin{bmatrix}
        0 & 1 \\
        0 & 0
    \end{bmatrix}$, then for matrix $A$, $a=d, c=0$ are true. (Problem 2) \par
    Therefore, for these 2 cases, $AB=BA$ true for all matrices $B$ when matrix $A$ has $a=d, b=0, c=0$.
\end{itemize}

\noindent 
\noindent 



\end{document} 