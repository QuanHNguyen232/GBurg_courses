\documentclass[10pt]{article}
\pagestyle{plain} \setlength{\textwidth}{12cm}
\setlength{\oddsidemargin}{2cm} \setlength{\evensidemargin}{2cm}
\setlength{\textheight}{23cm}
\usepackage{amssymb,latexsym,amsmath,amsthm,verbatim,calc}
\usepackage{graphicx,epsfig,epstopdf,amssymb,color,makeidx,float,caption,sidecap}

\begin{document}
\title{Math 212 written homework assignment week 3}
\author{Quan Nguyen}
\maketitle

\bigskip
\noindent
{\bf Acknowledgements and sources of help}:
\begin{itemize}
    \item Professor Ben Ken
    \item Sangya Lohani
    \item Julianna Mendez
    \item Ben (PLA)
    \item Tom Doan
\end{itemize}

\bigskip
\noindent

\section*{Problem 1}

\noindent Let $T$ the function that we get when we transform $B$, then $A$.\par
\noindent We want to show that $T$ is linear. Here is the definition: linear transformation $T$ is linear if:
\begin{itemize}
    \item $T(\Vec{u} + \Vec{v}) = T\Vec{u} + T\Vec{v}$
    \item $T(c\Vec{u}) = cT(\Vec{u})$
\end{itemize}

\noindent From the given formula: $T(\Vec{v}) = A(B\Vec{v})$ which equals to $ T\Vec{v} = AB\Vec{v}$.\par

\noindent For the first definition: $T(\Vec{u} + \Vec{v}) = T\Vec{u} + T\Vec{v}$, let $\Vec{u}$, $\Vec{v}$ are any two vectors.\par

\noindent Then, we have $B(\Vec{u} + \Vec{v})$, then $A[B(\Vec{u} + \Vec{v})]$ and use Theorem 5: $A(\Vec{u} + \Vec{v}) = A\Vec{u} + A\Vec{v}$:
\begin{align*}
    A[B(\Vec{u} + \Vec{v})]
    &= A[B\Vec{u} + B\Vec{v}] \\
    &= A(B\Vec{u}) + A(B\Vec{v}) \\
    &= T\Vec{u} + T\Vec{v}
\end{align*}
\par

\noindent For the second definition, let $\Vec{u}$ is any vector and $c$ is any scalar of $\Vec{u}$. \par
\noindent Suppose we have $B(c\Vec{u})$ and then $A[B(c\Vec{u})]$:
\begin{align*}
    A[B(c\Vec{u})] &= A[cB(\Vec{u})] \text{  (Theorem 5)}\\
    &= cA(B\Vec{u}) \\
    &= cT(\Vec{u})
\end{align*}

\noindent From those above, $T$ satisfies all two definitions of a linear, so $T$ is linear.

\section*{Problem 2}

\noindent Let $A\Vec{v} = \Vec{x}$, then we have $A(A\Vec{v}) = A\Vec{x}$.\par

\noindent For this question, it is easier to prove it is wrong by giving an example that $A\Vec{x} = \Vec{0}$ while $\Vec{x}\neq \Vec{0}.$

\noindent Suppose $n = 2$, so $A$ is a 2x2 matrix:
\begin{equation*}
    \begin{bmatrix}
        a & b \\
        c & d
    \end{bmatrix}
    \begin{bmatrix}
        x_1 \\
        x_2
    \end{bmatrix}
    =
    \begin{bmatrix}
        0 \\
        0
    \end{bmatrix}
\end{equation*}
\par

\noindent In this case, I will choose this matrix and this vector:
\begin{equation*}
    \begin{bmatrix}
        1 & -1 \\
        1 & -1
    \end{bmatrix}
    \begin{bmatrix}
        1 \\
        1
    \end{bmatrix}
    =
    \begin{bmatrix}
        0 \\
        0
    \end{bmatrix}
\end{equation*}

\noindent Then, we must find $\Vec{v}$ to verify if there exists $\Vec{x}$ with that $A$:
\begin{equation*}
    \begin{bmatrix}
        1 & -1 \\
        1 & -1
    \end{bmatrix}
    \begin{bmatrix}
        v_1 \\
        v_2
    \end{bmatrix}
    =
    \begin{bmatrix}
        1 \\
        1
    \end{bmatrix}
\end{equation*}

\noindent Written it as augmented matrix:
\begin{equation*}
    \begin{bmatrix}
        1 & -1 && 1 \\
        1 & -1 && 1
    \end{bmatrix}
    =
    \begin{bmatrix}
        1 & -1 && 1 \\
        0 & 0 && 0
    \end{bmatrix}
\end{equation*}

\noindent Now we convert it back to the system:
\begin{align*}
    v_1 - v_2 &= 1 \\
    v_2 &\text{ is free}
\end{align*}

\noindent Therefore, if we choose:
\begin{equation*}
    A = 
    \begin{bmatrix}
        1 & -1 \\
        1 & -1
    \end{bmatrix},
    \Vec{v} = 
    \begin{bmatrix}
        v_2 + 1 \\
        v_2
    \end{bmatrix}
    \text{ with any $v_2$}
\end{equation*}

\noindent We then have (suppose $v_2 = 2$):
\begin{equation*}
    A\Vec{v} = 
    \begin{bmatrix}
        1 & -1 \\
        1 & -1
    \end{bmatrix}
    \begin{bmatrix}
        3 \\
        2
    \end{bmatrix}
    =
    \begin{bmatrix}
        1 \\
        1
    \end{bmatrix}
\end{equation*}

\noindent However:
\begin{equation*}
    A(A\Vec{v}) =
    \begin{bmatrix}
        1 & -1 \\
        1 & -1
    \end{bmatrix}
    \begin{bmatrix}
        1 \\
        1
    \end{bmatrix}
    =
    \begin{bmatrix}
        0 \\
        0
    \end{bmatrix}
\end{equation*}\par
\bigskip
\noindent Thus, we can prove that $A\Vec{v}$ is not $\Vec{0}$, but $A(A\Vec{v}) = \Vec{0}$.

\end{document} 